\documentclass[a4paper]{extarticle}

\usepackage[12pt]{extsizes}
\usepackage{comment}
\usepackage{enumitem}
\usepackage{hyperref}
\usepackage{cmap}
\usepackage{tabu}
\usepackage{float}
\usepackage{listings}
\usepackage{mathptmx}
\usepackage{amsmath, amsfonts}
\usepackage{indentfirst}
\usepackage{graphicx}
\usepackage[T2A]{fontenc}
\usepackage[utf8]{inputenc}
\usepackage[english,russian]{babel}
\renewcommand{\labelitemi}{–}
\renewcommand{\labelenumii}{\theenumii}
\renewcommand{\theenumii}{\theenumi.\arabic{enumii}.}

\linespread{1.3}
\renewcommand{\rmdefault}{ftm}
\usepackage[left=20mm, top=15mm, right=15mm, bottom=15mm, nohead, footskip=10mm]{geometry}

% JavaScript
\lstdefinelanguage{JavaScript}{
  morekeywords={typeof, new, true, false, catch, function, return, null, catch, switch, var, if, in, while, do, else, case, break},
  morecomment=[s]{/*}{*/},
  morecomment=[l]//,
  morestring=[b]",
  morestring=[b]'
}

\begin{document}

\section{Экономическая часть}

\subsection{Обзор рынка программных продуктов}
Разрабатываемая подсистема визуализации аналитического отчета должна обеспечивать следующую функциональность:\par
\begin{itemize}
  \item возможность преобразовывать аналитический источник данных в удобный для передачи и чтения компонентом визуализации формат;
  \item возможность запрашивать у пользователя параметры визуализации (наименование визуализации, названия осей, тип визуализации);
  \item возможность объединения собранных данных в удобный для передачи и чтения компонентом визуализации формат;
  \item возможность построения объекта визуализации, который должен содержать:
    \begin{itemize}
    	\item наименование визуализации;
        \item легенду, указывающую описание к маркерам данных;
    	\item оси и подписи к ним, если выбрана столбчатая диаграмма или график;
        \item метки осей - категории, которые задают положение конкретных значений в ряде данных;
    \end{itemize}
  \item возможность интеграции объекта визуализации в программный комплекс «Web-Исполнение бюджета».
\end{itemize}\par
Представляемая подсистема создана для узкоспециализированной задачи и использования только в рамках программного продукта «Web-Исполнение бюджета».\par
При изучении рынка специализированного ПО было выявлено, что аналогичные «Web-Исполнению бюджета» системы так же используют подобный функционал. В большинстве этих систем используются комерческие приложения крупных компаний, таких, как Microsoft, Oracle, cluvio и других.\par
Oracle Data Visualization Cloud Service, Microsoft Excel Online и cluvio - это платные коммерческие приложения, которые требуют не разовую оплату продукта, а распространяются по системе подписки.

\subsection{Расчет себестоимости подсистемы}
Расчет себестоимости подсистемы определяется суммированием расходов по следующим пунктам:
\begin{itemize}
\item основная заработная плата;
\item размер социальных отчислений;
\item накладные расходы;
\item материальные затраты;
\item затраты на машинное время;
\item прочие затраты.
\end{itemize}
Далее производится расчет затрат по каждому пункту.\par

\subsubsection{Материальные затраты}
Материальные затраты – вид затрат, образующих себестоимость продукции, и представляющий расходы, осуществленные на приобретение необходимых для производства товаров (выполнения работ, оказания услуг) расходуемых материальных ресурсов.\par
В рамках данной разработки материальные затраты представлены расходами на бумагу, оплата доступа в Интернет и т.п.\par
Расходы на бумагу и прочее можно считать незначительными и пренебречь ими. Доступ в интернет предоставляется по тарифу 550 рублей в месяц. Разработка велась 3 месяца; таким образом, материальные затраты составили 550$\times$3=1650 (руб.).

\subsubsection{Основная заработная плата}
Размер основной заработной платы (ОЗП) определяется окладом инженера-программиста, количеством разработчиков и сроками разработки:\par
\begin{equation}
\label{form1}
	\text{ОЗП} = \text{СЗП}\times T
\end{equation}
где\par
СЗП – стандартная заработная плата (16500 руб.)\par
T – число месяцев разработки (3 месяцев).\par
\begin{equation}
\label{form2}
	\text{ОЗП} = 16500\times 3
\end{equation}\par
Размер  ОЗП = 49500 руб.

\subsubsection{Социальные отчисления}
Отчисления на социальные нужды производятся согласно Федеральному закону №212-ФЗ «О страховых взносах в Пенсионный фонд Российской Федерации, Фонд социального страхования Российской Федерации, Федеральный фонд обязательного медицинского страхования (редакция от 02.04.2014)», который устанавливает следующие тарифы страховых взносов:
\begin{itemize}
  \item Пенсионный фонд Российской Федерации - 22\%;
  \item Фонд социального страхования Российской Федерации - 2,9\%;
  \item Федеральный фонд обязательного медицинского страхования - 5,1\%;
\end{itemize}\par
Суммарный процент социальных отчислений $C_o$ составляет $22\% + 2.9\% + 5.1\%$. Соответственно, размер социальных отчислений предриятия на заработную плату равенн:
\begin{equation}
\label{form3}
	\text{О}_\text{соц} = \text{ЗП}\times C_o = \text{ЗП}\times 30\%
\end{equation}\par
Получаем, что $\text{О}_\text{соц}$ = 14859 (руб.)\par
Сумма затрат предприятия на заявленную заработную плату работников составляет:
\begin{equation}
\label{form4}
	\text{Р}_\text{ЗП}=\text{ЗП}\times(1+C_o)
\end{equation}

\begin{equation}
\label{form5}
	\text{Р}_\text{ЗП}=49500\times(1+0.3)
\end{equation}\par

\subsubsection{Затраты на машинное время}
Расходы на машинное время с учетом наличия 254 дней в году определяются по формуле:\par
\begin{equation}
\label{form6}
	\text{Мвр}=\text{Счмв}\times \text{ВР}_\text{м}
\end{equation}
где $\text{М}_\text{вр}$ – расходы на машинное время, руб.;\par
Счмв – стоимость одного часа машинного времени, руб./час;\par
ВРм – время использования ЭВМ для разработки программного продукта, час.\par
Стоимость одного часа машинного времени можно рассчитать по формуле:\par
\begin{equation}
\label{form7}
	\text{Счмв}=\frac{\text{Цк}}{\text{Ссл}\times 254\times \text{Вэд}}+\text{РЭ}
\end{equation}
где $\text{Ц}_\text{ро}$ – цена рабочего окружения (покупная цена программного и аппаратного обеспечения), руб;\par
Ссл – срок службы компьютера, лет;\par
Вэд – время эксплуатации ЭВМ в день, час;\par
Рэ – эксплуатационные расходы, руб./час.\par
Покупная цена программного и аппаратного обеспечения составляет сумму цены компьютера, цены операционной системы и цены среды разработки:
\begin{equation}
	\text{Ц}_\text{ро} = \text{Ц}_\text{к}+\text{Ц}_\text{OS}+\text{Ц}_\text{ср}
\end{equation}
где $\text{Ц}_\text{ро}$ – цена рабочего окружения;
$\text{Ц}_\text{к}$ – цена компьютера с учетом всех комплектующих (35000 руб.);
$\text{Ц}_\text{OS}$ – цена операционной системы (9000 руб.);
$\text{Ц}_\text{ср}$ – цена среды разработки (840 руб. в месяц $\times$ 3 месяца = 2520 руб.).
Эксплуатационные расходы Pэ вычисляются по формуле:\par
\begin{equation}
	\text{Ц}_\text{ро} = 35000+9000+2520=46520
\end{equation}
\begin{equation}
\label{form8}
	\text{Рэ}=\text{Сэ}\times \text{Р}=4\times 0.3\approx 1.2 \text{ руб./час}
\end{equation}
где Сэ – стоимость 1кВт$\times$ч электроэнергии, руб./кВт$\times$ч;\par
Р – суммарная потребляемая мощность вычислительной системы, кВт.\par
Рассчитаем стоимость одного часа машинного времени:\par
\begin{equation}
\label{form9}
	\text{Счмв}=\frac{46520}{8\times 246\times 3}+1.2 = 9.07\text{ руб./час}
\end{equation}\par
Возьмем среднее значение рабочих дней в месяц (21 день), время работы вычислительной машины равно 21$\times$3 = 63 (дня), то есть в часах время использования ЭВМ составит:\par
\begin{equation}
\label{form10}
	\text{ВРм}=63\times 8=288\text{ (ч.)}
\end{equation}
Следовательно, можно найти расходы на машинное время:
\begin{equation}
\label{form11}
	\text{Мвр}=\text{Счмв}\times\text{ВРм}=7.9\times 288 = 2296.8\text{ (руб.)}
\end{equation}
\subsubsection{Накладные расходы}
В накладных расходах учитываются расходы на управление, коммунальные услуги, аренду помещения. Величина накладных расходов определяется в процентах от основной заработной платы, составляет 30\%, т.е. 14850 рублей:
\begin{equation}
\label{form12}
	\text{Р}_\text{Н}=49500\times 0.3=14850\text{ (руб.)}
\end{equation}

\subsubsection{Прочие затраты}
Прочие затраты могут включать:
\begin{itemize}
\item командировочные расходы разработчика, связанные с процессом разработки (учитываются по нормативу);
\item коммерческие расходы, связанные с реализацией программного продукта (в размере 2 – 3\% от производственной себестоимости – суммы материальных затрат, заработной платы с отчислениями и стоимости машинного времени);
\item расходы на рекламу.
\end{itemize}\par
При расчете суммы общих затрат будем исходить из того, что командировочные расходы и расходы на рекламу отсутствуют, а величина коммерческих расходов составляет 2,5\% от производственной себестоимости программного продукта:
\begin{equation}
\label{form13}
	\text{ПР}=(\text{МЗ}+\text{ОЗП}+\text{О}_\text{есн}+\text{Мвр}+\text{Р}_\text{Н})\times 0.25
\end{equation}
$$\text{ПР}=(1650+49500+64350+2217.6+14850)\times 0.25=3314.19\text{ (руб.)}$$

\subsubsection{Определение себестоимости разрабатываемой подсистемы}
Себестоимость разработки программного средства представляет собой сумму затрат по экономическим элементам, приведенным в таблице 1.\par
\begin{table}[H]
\caption{Экономические элементы для расчета $\text{П}_\text{себ}$}
\centering
  \begin{tabular}{|c|c|}
  \hline
  Затраты & Сумма, руб. \\\hline
  Материальные затраты & 1650 \\\hline
  Основная заработная плата & 49500 \\\hline
  Единый социальные налог & 64350 \\\hline
  Затраты на машинное время & 2217.60 \\\hline
  Накладные расходы & 14850 \\\hline
  Итог: & 132567.60 \\
  \hline
  \end{tabular}
\end{table}\par
Таким образом, стоимость разработки составила 132567.60 (руб.).\par

\subsubsection{Определение цены разрабатываемой подсистемы}
Разрабатываемая подсистема попадает в следующую версию «Web-Исполнения бюджета». На данный момент заказ на неё оформлен 8 заказчиками.\par
Рассчитаем рекомендуемую цену:\par
\begin{equation}
\label{form9}
	\text{Цр}=\frac{\text{Себ}\times (1+\text{Нпр})+(\text{Себ}\times(1+\text{Нпр})-\text{НР})\times \text{НДС}}{8}
\end{equation}\par\vspace{1cm}
где Нпр – норматив прибыли (10\%);\par
НР – накладные расходы;\par
НДС – налог на добавленную стоимость (18\%);\par
Себ – себестоимость разработки.\par
1З – количество заказчиков;\par
\begin{equation}
\label{form9}
	\text{Цр}=\frac{132567.60\times (1+0.1)+(132567.60\times (1+0.1)-14850)\times 0.18}{8}=5929.7\text{(руб.)}
\end{equation}
Итак, рекомендуемая цена подсистемы равна 5929.7 (руб.).\par

\subsubsection{Оценка возможных результатов от использования}
Внедрение разработанной подсистемы позволяет расширить возможности «Web-Исполнения бюджета». Благодаря данной подсистеме увеличивается интеграция «Web-Исполнения бюджета» в схему Единого Регионального Бюджета, что является прямым продолжением политики ООО «НПО «Криста».\par
С точки зрения пользователя, использование данного модуля повышает качество анализа данных, не нарушая тем самым свободы взаимодействия с данными. Поскольку вызов подсистемы, построение визуализации по готовому Аналитическому источнику данных, осуществляется минимальным нажатием на кнопку, расположенную на панели кнопок интерфейса «Отчеты», использование подсистемы для пользователя представляет абсолютно тривиальную задачу, не требующую особых усилий и времени на обучение.

\end{document}