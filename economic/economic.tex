\documentclass[a4paper]{extarticle}

\usepackage[14pt]{extsizes}
\usepackage{comment}
\usepackage{enumitem}
\usepackage{hyperref}
\usepackage{cmap}
\usepackage{tabu}
\usepackage{float}
\usepackage{listings}
\usepackage{mathptmx}
\usepackage{amsmath, amsfonts}
\usepackage{titlesec}
\usepackage{indentfirst}
\usepackage{graphicx}
\usepackage{longtable}
\usepackage[usenames,dvipsnames]{color} % названия цветов
\renewcommand{\rmdefault}{ftm} % Times New Roman
\usepackage[T2A]{fontenc}
\usepackage{mathtext}
\usepackage[utf8]{inputenc}
\usepackage[russian,english]{babel}

\renewcommand{\labelitemi}{–}
\renewcommand{\labelenumii}{\theenumii}
\renewcommand{\theenumii}{\theenumi.\arabic{enumii}.}
\setlist{nolistsep}

\setcounter{secnumdepth}{4}
\titleformat{\paragraph}
{\normalfont\normalsize\bfseries}{\theparagraph}{1em}{}
\titlespacing*{\paragraph}
{0pt}{3.25ex plus 1ex minus .2ex}{1.5ex plus .2ex}

\usepackage{chngcntr}
\counterwithin{figure}{section}
\counterwithin{table}{section}
\renewcommand{\thefigure}{\thesection.\arabic{figure}} % Формат рисунка секция.номер
\renewcommand{\thetable}{\thesection.\arabic{table}} % Формат таблицы секция.номер
\numberwithin{equation}{section}

\usepackage[tableposition=top]{caption}
\captionsetup[table]{
  labelsep = newline,
  textfont = sc, 
  name = TABLE, 
  justification=justified,
  singlelinecheck=false,%%%%%%% a single line is centered by default
  labelsep=colon,%%%%%%
  skip = \medskipamount}
\usepackage{subcaption}
\DeclareCaptionLabelFormat{gostfigure}{Рисунок #2}
\DeclareCaptionLabelFormat{gosttable}{Таблица #2}
\DeclareCaptionLabelSeparator{gost}{~---~}
\captionsetup{labelsep=gost}
\captionsetup[figure]{labelformat=gostfigure}
\captionsetup[table]{labelformat=gosttable}
\renewcommand{\thesubfigure}{\asbuk{subfigure}}

\newcommand{\TT}[2]{\text{\textit{#1}}_\text{\textit{#2}}}

\linespread{1.3}
\parindent=1.25cm
\setlist[enumerate,itemize]{itemindent=1.7cm, leftmargin=0cm}
\renewcommand{\rmdefault}{ftm}
\usepackage[left=20mm, top=15mm, right=15mm, bottom=15mm, nohead, footskip=10mm]{geometry}
\setlength{\parskip}{1.5pt}
% \usepakage{ragged2e}
% \justifying
\sloppy
\hyphenpenalty=10000
\clubpenalty=10000
\widowpenalty=10000
\setcounter{page}{8}

\lstset{
	basicstyle=\small\ttfamily,
    columns=fullflexible,
    % frame=single,
    breaklines=true,
}

% JavaScript
\lstdefinelanguage{JavaScript}{
  morekeywords={typeof, new, true, false, catch, function, return, null, catch, switch, var, if, in, while, do, else, case, break},
  morecomment=[s]{/*}{*/},
  morecomment=[l]//,
  morestring=[b]",
  morestring=[b]'
}

\begin{document}

\section{Экономическая часть}

\subsection{Обзор рынка программных продуктов}
Разрабатываемая подсистема визуализации аналитического отчета должна обеспечивать следующую функциональность:\par
\begin{itemize}
  \item возможность преобразовывать аналитический источник данных в удобный для передачи и чтения компонентом визуализации формат;
  \item возможность запрашивать у пользователя параметры визуализации (наименование визуализации, названия осей, тип визуализации);
  \item возможность объединения собранных данных в удобный для передачи и чтения компонентом визуализации формат;
  \item возможность построения объекта визуализации, который должен содержать:
    \begin{itemize}
    	\item наименование визуализации;
        \item легенду, указывающую описание к маркерам данных;
    	\item оси и подписи к ним, если выбрана столбчатая диаграмма или график;
        \item метки осей - категории, которые задают положение конкретных значений в ряде данных;
    \end{itemize}
  \item возможность интеграции объекта визуализации в программный комплекс "Web-Исполнение бюджета".
\end{itemize}\par
Представляемая подсистема создана для узкоспециализированной задачи и использования только в рамках программного продукта "Web-Исполнение бюджета".\par
При изучении рынка специализированного ПО было выявлено, что аналогичные "Web-Исполнению бюджета" системы так же используют подобный функционал. В большинстве этих систем используются комерческие приложения крупных компаний, таких, как Microsoft, Oracle, cluvio и других.\par
Oracle Data Visualization Cloud Service, Microsoft Excel Online и cluvio - это платные коммерческие приложения, которые требуют не разовую оплату продукта, а распространяются по системе подписки.

\subsection{Расчет себестоимости подсистемы}
Расчет себестоимости подсистемы определяется суммированием расходов по следующим пунктам:
\begin{itemize}
\item основная заработная плата;
\item размер социальных отчислений;
\item накладные расходы;
\item материальные затраты;
\item затраты на машинное время;
\item прочие затраты.
\end{itemize}
Далее производится расчет затрат по каждому пункту.\par

\subsubsection{Материальные затраты}
Материальные затраты – вид затрат, образующих себестоимость продукции, и представляющий расходы, осуществленные на приобретение необходимых для производства товаров (выполнения работ, оказания услуг) расходуемых материальных ресурсов.\par
В рамках данной разработки материальные затраты представлены расходами на бумагу, оплата доступа в Интернет и т.п.\par
Расходы на бумагу и прочее можно считать незначительными и пренебречь ими. Доступ в интернет предоставляется по тарифу 550 руб. в месяц. Разработка велась 3 месяца; таким образом, материальные затраты составили 550$\times$3=1650 руб.

\subsubsection{Основная заработная плата}
Размер основной заработной платы (ОЗП) определяется окладом инженера-программиста, количеством разработчиков и сроками разработки:\par
\begin{equation}
\label{form1}
	\text{ОЗП} = \text{СЗП}\times T
\end{equation}\newline
где\par
СЗП – стандартная заработная плата (16500 руб.)\par
T – число месяцев разработки (3 месяцев).\par
\begin{equation}
\label{form2}
	\text{ОЗП} = 16500\times 3
\end{equation}\par
Размер  ОЗП = 49500 руб.

\subsubsection{Социальные отчисления}
Отчисления на социальные нужды производятся согласно Федеральному закону №212-ФЗ "О страховых взносах в Пенсионный фонд Российской Федерации, Фонд социального страхования Российской Федерации, Федеральный фонд обязательного медицинского страхования (редакция от 02.04.2014)", который устанавливает следующие тарифы страховых взносов:
\begin{itemize}
  \item Пенсионный фонд Российской Федерации - 22\%;
  \item Фонд социального страхования Российской Федерации - 2,9\%;
  \item Федеральный фонд обязательного медицинского страхования - 5,1\%;
\end{itemize}\par
Суммарный процент социальных отчислений $C_o$ составляет $22\% + 2,9\% + 5,1\%$. Соответственно, размер социальных отчислений предриятия на заработную плату равенн:
\begin{equation}
\label{form3}
	\text{О}_\text{соц} = \text{ЗП}\times C_o = \text{ЗП}\times 30\%
\end{equation}\par
Получаем, что $\text{О}_\text{соц}$ = 14850 руб.\par
Сумма затрат предприятия на заявленную заработную плату работников составляет:
\begin{equation}
\label{form4}
	\text{Р}_\text{ЗП}=\text{ЗП}\times(1+C_o)
\end{equation}

\begin{equation}
\label{form5}
	\text{Р}_\text{ЗП}=49500\times(1+0,3)
\end{equation}\par

\subsubsection{Затраты на машинное время}
Расходы на машинное время с учетом наличия 246 дней в году определяются по формуле:\par
\begin{equation}
\label{form6}
	\text{Мвр}=\text{Счмв}\times \text{ВР}_\text{м}
\end{equation}
где $\text{М}_\text{вр}$ – расходы на машинное время, руб.;\par
Счмв – стоимость одного часа машинного времени, руб./час;\par
ВРм – время использования ЭВМ для разработки программного продукта, час.\par
Стоимость одного часа машинного времени можно рассчитать по формуле:\par
\begin{equation}
\label{form7}
	\text{Счмв}=\frac{\text{Цк}}{\text{Ссл}\times 246\times \text{Вэд}}+\text{РЭ}
\end{equation}
где $\text{Ц}_\text{ро}$ – цена рабочего окружения (покупная цена программного и аппаратного обеспечения);\par
Ссл – срок службы компьютера, лет;\par
Вэд – время эксплуатации ЭВМ в день, час;\par
Рэ – эксплуатационные расходы, руб./час.\par
Покупная цена программного и аппаратного обеспечения составляет сумму цены компьютера, цены операционной системы и цены среды разработки:
\begin{equation}
	\text{Ц}_\text{ро} = \text{Ц}_\text{к}+\text{Ц}_\text{OS}+\text{Ц}_\text{ср}
\end{equation}
где $\text{Ц}_\text{ро}$ – цена рабочего окружения;\par
$\text{Ц}_\text{к}$ – цена компьютера с учетом всех комплектующих (35000 руб.);\par
$\text{Ц}_\text{OS}$ – цена операционной системы (9000 руб.);\par
$\text{Ц}_\text{ср}$ – цена среды разработки (840 руб. в месяц $\times$ 3 месяца = 2520 руб.).\par
Эксплуатационные расходы Pэ вычисляются по формуле:\par
\begin{equation}
	\text{Ц}_\text{ро} = 35000+9000+2520=46520
\end{equation}
\begin{equation}
\label{form8}
	\text{Рэ}=\text{Сэ}\times \text{Р}=4\times 0.3\approx 1.2 \text{ руб./час}
\end{equation}
где Сэ – стоимость 1кВт$\times$ч электроэнергии, руб./кВт$\times$ч;\par
Р – суммарная потребляемая мощность вычислительной системы, кВт.\par
Рассчитаем стоимость одного часа машинного времени:\par
\begin{equation}
\label{form9}
	\text{Счмв}=\frac{46520}{8\times 246\times 3}+1.2 = 9,07\text{ руб./час}
\end{equation}\par
Возьмем среднее значение рабочих дней в месяц (21 день), время работы вычислительной машины равно 21$\times$3 = 63 (дня), то есть в часах время использования ЭВМ составит:\par
\begin{equation}
\label{form10}
	\text{ВРм}=63\times 8=288\text{ (ч.)}
\end{equation}
Следовательно, можно найти расходы на машинное время:
\begin{equation}
\label{form11}
	\text{Мвр}=\text{Счмв}\times\text{ВРм}=9,07\times 288 = 2612,16\text{ руб.}
\end{equation}
\subsubsection{Накладные расходы}
В накладных расходах учитываются расходы на управление, коммунальные услуги, аренду помещения. Величина накладных расходов определяется в процентах от основной заработной платы, составляет 30\%, т.е. 14850 руб.:
\begin{equation}
\label{form12}
	\text{Р}_\text{Н}=49500\times 0,3=14850\text{ руб.}
\end{equation}

\subsubsection{Прочие затраты}
Прочие затраты могут включать:
\begin{itemize}
\item командировочные расходы разработчика, связанные с процессом разработки (учитываются по нормативу);
\item коммерческие расходы, связанные с реализацией программного продукта (в размере 2 – 3\% от производственной себестоимости – суммы материальных затрат, заработной платы с отчислениями и стоимости машинного времени);
\item расходы на рекламу.
\end{itemize}\par
При расчете суммы общих затрат будем исходить из того, что командировочные расходы и расходы на рекламу отсутствуют, а величина коммерческих расходов составляет 2,5\% от производственной себестоимости программного продукта:
\begin{equation}
\label{form13}
	\text{ПР}=(\text{МЗ}+\text{ОЗП}+\text{О}_\text{есн}+\text{Мвр}+\text{Р}_\text{Н})\times 0,025
\end{equation}
$$\text{ПР}=(1650+49500+14850+2612,16+14850)\times 0,025=2086,55\text{ руб.}$$

\subsubsection{Определение себестоимости разрабатываемой подсистемы}
Себестоимость разработки программного средства представляет собой сумму затрат по экономическим элементам, приведенным в таблице 1.\par
\begin{table}[H]
\caption{Экономические элементы для расчета $\text{П}_\text{себ}$}
  \begin{tabular}{|c|c|}
  \hline
  Затраты & Сумма, руб. \\\hline
  Материальные затраты & 1650 \\\hline
  Основная заработная плата & 49500 \\\hline
  Социальные отчисления & 14850 \\\hline
  Затраты на машинное время & 2612,16 \\\hline
  Накладные расходы & 14850 \\\hline
  Прочие затраты & 2086,55 \\\hline
  Итог: & 85548,71 \\
  \hline
  \end{tabular}
\end{table}\par
Таким образом, стоимость разработки составила 85548,71 руб.\par

\subsection{Оценка эффекта от продажи подсистемы}

\subsubsection{Расчет прибыли от продажи одной копии подсистемы}
Разрабатываемая подсистема попадает в следующую версию программного комплекса "Web-Исполнения бюджета". На данный момент заказ на неё оформлен 8 заказчиками:\par
\begin{table}[H]
\caption{Заказчики, согласившиеся включить в следющее обновление разработанную подсистему}
  \begin{tabular}{|c|c|}
  \hline
  № & Наименование \\\hline
  1 & Долгопрудный (ФУ) Московская обл. \\\hline
  2 & Ленинский район (ФУ) Московская обл. \\\hline
  3 & Можайский район (ФКУ) Московская обл. \\\hline
  4 & Орехово-Зуево (ФУ ГО) Московская обл. \\\hline
  5 & Подольск (ФУ) Московская обл. \\\hline
  6 & Пушкинский р-н (КФНП) Московская обл. \\\hline
  7 & Фрязино (ФУ) Московская обл. \\\hline
  8 & Электросталь (ФУ) Московская обл. \\\hline
  \end{tabular}
\end{table}\par

Расчет прибыли от продажи одной копии подсистемы осуществляется по следующей формуле:
\begin{equation}
\label{form13}
	\text{П}_\text{ед}=\text{Ц}-\text{НДС}-\frac{\text{З}_\text{р}}{N}
\end{equation}
где $\text{П}_\text{ед}$ - прибыли от продажи одной копии;\par
$\text{Ц}$ - цена одной копии;\par
$\text{НДС}$ - налог на добавленную стоимость;\par
$\text{З}_\text{р}$ - затраты на реализацию;\par
N - количество заказчиков.\par
За цену одной копии возьмем 20\% от себестоимости подсистемы, то есть $85548,71\times 0,2 = 17000$ руб.
Расчитаем НДС:
\begin{equation}
\label{form14}
	\text{НДС}=\frac{\text{Ц}\times \text{\%НДС}}{100\%+18\%}=\frac{17000\times 18\%}{100\%+18\%}=2609,9 \text{ руб.}
\end{equation}\par
Расчитаем затраты на реализацию, которые составляют 10\% от себестоимости разработки:
\begin{equation}
\label{form15}
	85548,71\times 0,1 = 8554,87 \text{ руб.}
\end{equation}\par
Таким образом прибыль от продажи одной копии будет составлять:
\begin{equation}
\label{form16}
	\text{П}_\text{ед}=17000-2609,9-\frac{85548,71+8554,87}{8}=2627,15\text{ руб.}
\end{equation}
Затраты на разработку и реализацию составляют 85548,71 и 8554,87 руб. соответственно. Доход от продажи одной копии подсистемы составляет:\par
\begin{equation}
\label{form17}
	\text{Ц}-\text{\%НДС} = 17000-2609,9 = 14390,1 \text{ руб.}
\end{equation}\par
Рассчитаем количество копий, которые необходимо продать для окупаемости затрат:\par
\begin{equation}
\label{form18}
	\frac{85548,71+8554,87}{14390,1}=7 \text{ (копий)}
\end{equation}
После 7 проданных заказчикам копий, дальнейшие продажи будут приносить прибыль.

\subsubsection{Экономический эффект от внедрения подсистемы}
Подсистема визуализации аналитических отчетов позволит пользователям "Web-Исполнение бюджета" сократить в несколько раз время, затрачиваемое на обновление и сбор данных об организациях.\par
Критерием эффективности создания и внедрения новых средств автоматизации является ожидаемый экономический эффект. Он определяется по формуле:\par
\begin{equation}
\label{form19}
	\text{Э}=\text{Э}_\text{р}-\text{Е}_\text{н}\times \text{К}_\text{п}
\end{equation}\par
где $\text{Э}_\text{р}$ - годовая экономия;\par
$\text{Е}_\text{н}$ - нормативный коэффициент (0,15);\par
$\text{К}_\text{п}$ - капитальные затраты на проектирование и внедрение, включая первоначальную стоимость программы.\par
К капитальным затратам необходимо отнести себестоимость подсистемы 85548,71 руб.\par
\begin{equation}
\label{form20}
	\text{Э}_\text{р}=(\text{Р}_\text{1}-\text{Р}_\text{2})+\Delta\text{Р}_\text{п}
\end{equation}\par
Эксплуатационные расходы не снижаются, поэтому при расчете экономии основным показателем будет экономия от повышения производительности труда($\Delta\text{Р}_\text{п}$).\par
Если учитывать среднюю заработную плату сотрудников федеральных органов государственной власти, работающих с программным комплексом "Web-Исполнением бюджета" - 30 000 руб./месяц, то повышение производительности труда $\text{Р}_i$ (\%) определяется по формуле:\par
\begin{equation}
\label{form21}
	\text{Р}_i=(\frac{\Delta T_j}{F_j-\Delta T_j})\times 100\%
\end{equation}\par
где $F_j$ - время, которое планировалось пользователем для выполнения работы j-го вида до внедрения программы;\par
$\Delta T_j$ - экономия времени с применением доработки.\par
Если учесть, что сотрудник федерального органа государственной власти обычно тратит 4 часа в день на анализ табличных отчетов, то в месяц у сотрудника уходит 84 часа. Визуализация аналитического отчета увеличит наглядность табличных данных, поэтому преположим, что сотрудник в день станет тратить 1 час на анализ данных, то есть 21 часа в неделю.\par
Экономия в процентах для подсистемы вычисляется по формуле:
\begin{equation}
\label{form22}
	\text{Р}_i=(\frac{63}{84-63})\times 100\%=300\%
\end{equation}\par
Можно заключить, что при использовании доработки сотрудник стал гораздо эффективнее тратить рабочее время, соответственно у него освободилось время на реализацию других задач. Из этого можно сделать вывод, что обеим сторонам будет выгодна данная доработка.

\end{document}

















