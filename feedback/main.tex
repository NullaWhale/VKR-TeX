\documentclass[a4paper]{extarticle}

\usepackage[14pt]{extsizes}
\usepackage{comment}
\usepackage{enumitem}
\usepackage{hyperref}
\usepackage{cmap}
\usepackage{tabu}
\usepackage{float}
\usepackage{listings}
\usepackage{mathptmx}
\usepackage{amsmath, amsfonts}
\usepackage{indentfirst}
\usepackage{graphicx}
\usepackage[T2A]{fontenc}
\usepackage[utf8]{inputenc}
\usepackage[english,russian]{babel}
\renewcommand{\labelitemi}{–}
\renewcommand{\labelenumii}{\theenumii}
\renewcommand{\theenumii}{\theenumi.\arabic{enumii}.}

\linespread{1.3}
\renewcommand{\rmdefault}{ftm}
\usepackage[left=20mm, top=15mm, right=15mm, bottom=15mm, nohead, footskip=10mm]{geometry}
\pagestyle{empty}

% JavaScript
\lstdefinelanguage{JavaScript}{
  morekeywords={typeof, new, true, false, catch, function, return, null, catch, switch, var, if, in, while, do, else, case, break},
  morecomment=[s]{/*}{*/},
  morecomment=[l]//,
  morestring=[b]",
  morestring=[b]'
}

\begin{document}

{
\centering
\large ОТЗЫВ\par
\large на выпускную квалификационную работу\par
\large «Разработка подсистемы визуализации аналитического отчета в системе бюджетирования»\par
\large студента группы ИПБ-13\par
\large Ляпушкина Никиты Алексеевича\par\vspace{1cm}
}

Подсистема визуализации аналитического отчета в системе бюджетирования, разрабатываемая Ляпушкиным Н.А., является частью программного комплекса «Web-Исполнение Бюджета». Программный комплекс «Web-Исполнение Бюджета» обеспечивает ведения бюджетной отчетности и формирование аналитических форм отчетности на всех этапах исполнения бюджета РФ.\par
В ходе выполненяи выпускной квалификацонной работы Ляпушкиным Н.А. была изучена предметная область, рассмотрена структура программного комплекса «Web-Исполнение Бюджета», определено место разрабатываемых модулей в структуре комплекса, проанализированы различные способы интеграции разрабатываемой подсистемы, определены требования к разрабатываемой подсистеме, подготовлена проектная и техническая документация, выполнено экономическое обоснование разработки, разработан компонент визуализации, позволяющий отображать результаты визуализации в виде графика или диаграммы через web-интерфейс.\par
Работа выполнена на достаточно высоком профессиональном уровне.
К недостаткам можно отнести:
\begin{itemize}
\item отсутствие ссылок на некоторые внешние источники, указанные в тексте работы;
\item техническое задание выполнено не по стандарту ГОСТ, добавлены пункты, отсутствующие в стандарте.
\end{itemize}\par
Во время выполнения выпускной квалификационной работы Ляпушкин Н.А. проявил самостоятельность и ответственность. В работе показал отличные теоретические знания и практические навыки, стремление совершенствовать свои знания и приобрести опыт для самостоятельной деятельности.\par
Считаю, что Ляпушкин Н.А. заслуживает присвоения степени бакалавр по направлению 09.03.04 «Программная инженерия», а работа – оценки «хорошо».\par
\vspace{1cm}
\begin{flushright}
\begin{tabular}{p{.8\textwidth}}
  Руководитель, \hfill \\
  к.т.н., доцент каф. МПО ЭВС \hrulefill С.В. Маврин 
\end{tabular}
\end{flushright}


\end{document}