\documentclass[a4paper, 12pt]{report}
\usepackage[utf8]{inputenc}
\usepackage[T2A]{fontenc}
\usepackage[russian]{babel}
\usepackage[left=20mm, top=20mm, right=20mm, bottom=20mm, nohead, footskip=10mm]{geometry}
\parindent=1.25cm

\usepackage{enumitem}
\setlist[enumerate,itemize]{itemindent=1.25cm}
\setlist{nolistsep}
\renewcommand{\labelitemi}{-}

\begin{document}
\begin{center}
	\Large Речь
\end{center}\par
\begin{enumerate}[label=\textbf{\arabic*})]
	\item \textit{(Вступительный слайд)} Добрый день уважаемая комиссия! Вашему вниманию предоставляется выпускная квалификационная работа на тему "Разработка подсистемы визуализации аналитических отчетов в системе бюджетирования".
	\item \textit{(Бюджетирование)} Разработкой представляемой ВКР я занимался в научно-производственном объединении Криста в отделе разработки ПК "Web-Исполнение", которое работает над автоматизацией процесса бюджетирования. Бюджетирование - это производственно-финансовое планирование деятельности путем составления общего бюджета и бюджетов отдельных объектов.
	\item \textit{(Визуализация информации)} В следствии роста документного потока возникает потребность сокращения текста, свертывания информации или данных в более компактную структуру для улучшения наглядности. Так как ПК "Web-Исполнение бюджета" оперирует значительным объемом бюджетных данных, на основании которых строятся аналитические отчеты, представляемые в табличном виде, появляется потребность в более наглядном представлении этих данных. Было принято решение разработать подсистему визуализации аналитических отчетов.
	\item \textit{(Цели и задачи)} Целью написания ВКР является повышение наглядности анализа данных в ПК "Web-Исполнения бюджета". Также передо мной стояли следующие задачи, которые необходимо было решить:
		\begin{itemize}
			\item задание параметров визуализации;
			\item конвертирование аналитического в формат удобный для передачи;
			\item построение визуализации с использованием пользовательских параметров и аналитического источника;
			\item встраивание разработанных компонентов подсистемы в ПК "Web-Исполнение бюджета".
		\end{itemize}
	\item \textit{(Требования к типам визуализаций)} 
	\item \textit{(Используемый инструментарий)}Для разработки подсистемы визуализации аналитических отчетов были использован следующий инструментарий:
		\begin{itemize}
			\item Интеграция подсистемы в ПК "Web-Исполнение бюджета" производилась с помощью языков программирования JAVA и JavaScript, так как платформа ПК написана на этих языках;
			\item В качестве формата передачи данных был использован текстовый формат JSON;
			\item Для построения визуализации аналитических данных была использована свободная JavaScript-библиотека D3.js.
		\end{itemize}
	\item \textit{(Интерфейс «Аналитические источники данных»)} Для построения визуализации 
\end{enumerate}

\end{document}
